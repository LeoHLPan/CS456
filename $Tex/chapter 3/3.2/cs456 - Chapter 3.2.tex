\documentclass{article}

\usepackage[utf8]{inputenc}
\usepackage{fullpage}
\usepackage{times}
\usepackage{tcolorbox}
\usepackage{enumitem}
\usepackage{multicol}
\usepackage{bookmark}

\bookmarksetup{
  numbered, 
  open,
}

\setlist{itemsep=2pt}
\renewcommand{\thesection}{Chapter 3.\arabic{section}}
\renewcommand{\thesubsection}{3.\arabic{section}.\arabic{subsection}}
\setcounter{section}{1}

\begin{document}

\noindent
{CS 456 \hfill Hao Pan}\\
{Salahuddin, Mohammad}\\
{Spring 2018}

%%%%%%%%%%%%%%%%%%%%%%%%%%%%%%%%%%%

\begin{center}
\section{Multiplexing and Demultiplexing}
\noindent
{\hfill 12/06/2018 [T]}
\end{center}

\subsection{Overview}

\begin{itemize}
\item When \emph{multiplexing at the sender}, data must be handled from multiple sockets, and a transport header must be added, which is used later for demultiplexing.
\item When \emph{demultiplexing at the receiver}, the header info is used to deliver the received segments to the correct socket.
\end{itemize}

%%%%%%%%%%%%%%%%%%%%%%%%%%%%%%%%%%%
\subsection{How Demultiplexing Works}

\begin{itemize}
\item The host receives IP datagrams. Each datagram has a source and destination \emph{IP address} and carries \emph{one transport-layer segment}. They also have a source and destination \emph{port number}.
\item The host uses the IP address and port numbers to send the segment to the appropriate socket.
\item In a {\bf connectionless demux}, a datagram with the \emph{same destination port number} will be sent to the \emph{same socket} regardless of source info.
\item See example of a connectionless demux on slide 3-11.
\item In a {\bf connection-oriented demux}, \emph{all four header values} are used to determine destination socket.
\item See example of a connectionless demux on slide 3.13.
\end{itemize}


























\end{document}