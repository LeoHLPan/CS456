\documentclass{article}

\usepackage[utf8]{inputenc}
\usepackage{fullpage}
\usepackage{times}
\usepackage{tcolorbox}
\usepackage{enumitem}
\usepackage{multicol}
\usepackage{bookmark}

\usepackage[]{algorithm2e}

\bookmarksetup{
  numbered, 
  open,
}

\setlist{itemsep=2pt}
\renewcommand{\thesection}{Chapter 5.\arabic{section}}
\renewcommand{\thesubsection}{5.\arabic{section}.\arabic{subsection}}
\setcounter{section}{3}

\begin{document}

\noindent
{CS 456 \hfill Hao Pan}\\
{Salahuddin, Mohammad}\\
{Spring 2018}

%%%%%%%%%%%%%%%%%%%%%%%%%%%%%%%%%%%

\begin{center}
\section{Routing Among the ISPs: BGP}
\noindent
\end{center}

\subsection{Internet Inter-AS Routing: BGP}

\begin{itemize}
\item {\bf Border Gateway Protocol (BGP)} is the inter-domain routing protocol. It is \emph{"the blue that holds the Internet together"}.
\item BGP provides each AS a means to:
\begin{itemize}
\item {\bf External BGP (eBGP)}: to obtain subnet reachability information from neighbouring ASes.
\item {\bf Internal BGP (iBGP)}: to propagate reachability information to all AS-internal routers.
\item to determine "good" routes to other networks based on the reachability information and policy.
\end{itemize}
\item BGP allows a subnet to advertise its existence to the rest of the Internet.
\item Note that gateway routers run both eBGP and iBGP protocols.
\item In a {\bf BGP session}, two BGP routers (peers) would exchange BGP essages over a semi-permanent TCP connection. They advertise paths to different destination network prefixes. Thus, BGP is a \emph{"path vector" protocol}.
\item When an AS gateway router advertises a path to another AS gateway router, it promises that it will forward datagrams via that path.
\item The advertised network prefix includes BGP attributes such that the "route" = prefix + attributes.
\item Two important attributes are:
\begin{itemize}
\item {\bf AS-PATH}: a list of ASes which the prefix advertisement passed.
\item {\bf NEXT-HOP}: the IP address of the router that begins the AS-PATH.
\end{itemize}
\item In {\bf policy-based routing}, gateways receiving route advertisements use {\bf import policy} to accept or decline a path (an example policy is to never route through AS Y). AS policies also determine whether to advertise a path to other neighbouring ASes.
\item See examples of eBGP and iBGP in action on {\bf slide 5-45} and {\bf slide5-46}.
\item BGP messages can be exchanged between peers over a TCP connection.
\item BGP messages include:
\begin{itemize}
\item {\bf OPEN}: Opens a TCP connection to a remote BGP peer. Also used to authenticate OPEN messages sent by a peer.
\item {\bf UPDATE}: Advetises a new path or withdraws an old one.
\item {\bf KEEPALIVE}: Keeps a connection alive when there are no UPDATES. 
\item {\bf NOTIFICATION}: Reports errors in previous messages. Also used to close a connection.

\clearpage

\end{itemize}
\item A router may learn about more than one route to the destination AS. In such a case, a route is selected based on:
\begin{enumerate}
\item Local preference value attribute: policy decision.
\item Shortest AS-PATH.
\item Closest NEXT-HOP router: hot potato routing.
\item Additional criteria.
\end{enumerate}
\item {\bf Hot potato routing} chooses the local gateway that has the least intra-domain cost. It does not take into account the inter-domain cost. See an example of this in action on {\bf slide 5-51}.
\item If an ISP only wants to route traffic to and from its customer networks and does not want to carry traffic between other ISPs, it can \emph{choose not to advertise} the path to its neighbour. See an example on {\bf slide 5-52 to 5-53}.
\end{itemize}

%%%%%%%%%%%%%%%%%%%%%%%%%%%%%%%%%%%
\subsection{Why Split Between Intra- and Inter-AS Routing}

\begin{itemize}
\item {\bf Policies}:
\begin{itemize}
\item Inter-AS: Admins want control over how its traffic is routed and who goes through its net.
\item Intra-AS: Has a single admin, so no policy decisions are needed.
\end{itemize}
\item {\bf Scale}: Hierarchial routing saves table size and reduces update traffic.
\item {\bf Performance}:
\begin{itemize}
\item Intra-AS focuses on performance.
\item Inter-AS allows policies to sometimes take precedence over performance.
\end{itemize}
\end{itemize}





























\end{document}