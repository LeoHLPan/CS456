\documentclass{article}

\usepackage[utf8]{inputenc}
\usepackage{fullpage}
\usepackage{times}
\usepackage{tcolorbox}
\usepackage{enumitem}
\usepackage{multicol}
\usepackage{bookmark}

\usepackage[]{algorithm2e}

\bookmarksetup{
  numbered, 
  open,
}

\setlist{itemsep=2pt}
\renewcommand{\thesection}{Chapter 5.\arabic{section}}
\renewcommand{\thesubsection}{5.\arabic{section}.\arabic{subsection}}
\setcounter{section}{1}

\begin{document}

\noindent
{CS 456 \hfill Hao Pan}\\
{Salahuddin, Mohammad}\\
{Spring 2018}

%%%%%%%%%%%%%%%%%%%%%%%%%%%%%%%%%%%

\begin{center}
\section{Routing Protcols}
\noindent
\end{center}

\subsection{Overview}

\begin{itemize}
\item The goal of {\bf routing protocols} is to determine \emph{"good"} paths/routes from a sending host to the receiving host, through a network of routers.
\item A {\bf path} is a sequence of routers which packets will travel through while going from the source host to the destination host.
\item Whether or not a path is \emph{good} can be define in different ways, such as having the least cost, is the fasted, or is the least congested.
\item See an example of graph abstraction from a network on {\bf slide 5-9}.
\item The \emph{cost} of a path could be defined in different ways depending on preference. The cost could always be 1 (meaning shortest path wins), inversely related to bandwidth, or inversely related to congestion (take speed into account).
\end{itemize}

%%%%%%%%%%%%%%%%%%%%%%%%%%%%%%%%%%%
\subsection{Routing Algorithm Classifications}

\begin{itemize}
\item Routers can have \emph{global} or \emph{decentralized} information:
\begin{itemize}
\item {\bf Global}: All routers have complete a complete topology and the link cost info. This is the kind of information that \emph{"link state" algorithms} provide.
\item {\bf Decentralized}: Routers only know their physically-connected neighbours and the link cost to those neighbours. It is an iterative process where they exchange info with neighbours. This is the info that \emph{"distance vector" algorithms} provide.
\end{itemize}
\item Router information can be \emph{static} or \emph{dynamic}:
\begin{itemize}
\item {\bf Static} routes change slowly over time.
\item {\bf Dynamic} routes change more quickly in response to changes to link costs. They also update periodically.
\end{itemize}
\end{itemize}

%%%%%%%%%%%%%%%%%%%%%%%%%%%%%%%%%%%
\subsection{A Link-State Routing Algorithm}

\begin{itemize}
\item {\bf Dijkstra's algorithm} is commonly used to compute the net topology and link costs of a network.
\item It is guaranteed to compute the least cost paths from one node to all other nodes.
\item Using Dijkstra's algorithm, we know that after k iterations, we have the least cost paths to k destinations.
\item See pseudocode for Dijkstra's algorithm on {\bf slide 5-14}. See examples of it in action on {\bf slide 5-15} to {\bf slide 5-17}.
\item Dijkstra's algorithm has a complexity of $O(n^2)$, where $n$ is the number of nodes. A more efficient algorithm can reduce the complexity to $O(n\log n)$.
\end{itemize}

%%%%%%%%%%%%%%%%%%%%%%%%%%%%%%%%%%%
\subsection{Distance Vector Algorithm}

\begin{itemize}
\item The {\bf Bellman-Ford equation} (dynamic programming) is often used.
\item The algorithm can be found on {\bf slide 5-20}.
\item The algorithm essentially uses dynamic programming to compute the least cost path by checking the cost to reach each of the target node's neighbours.
\item In other words, suppose we are trying to find the shortest path from node $x$ to node $y$. Suppose $y$ has neighbours $v$. The Bellman-Ford algorithm will find the shortest path from $x$ to each $v$, then use that to compute the smallest of $D(x,v) + D(v,y)$. Do note that it will recursively go back by checking the neighbours of $v$ when looking for the shortest path from $x$ to $v$.
\item An example can be seen on {\bf slide 5-21}.
\item The key idea of this algorithm is that from time-to-time, each node will send its own distance vector (DV) estimate to its neighbours. When a node receives new DV estimates, it will update its own DV using bellman-ford.
\item The process is iterative, and each iteration is caused by either \emph{a change in local link costs} or \emph{a DV update message from a neighbour}. Note that each node only notifies neighbours if its DV changes, which causes their neighbours to notify their neighbours if there is change.
\item See examples of this algorithm in action as well as the related node tables on {\bf slides 5-25} and {\bf slide 5-26}.
\item If a link cost changes, a node will, when it detects the link cost change, update its own routing info and recalculate its distance vectors. If the DV changes, it will notify its neighbours.
\item Note that \emph{bad news travels slow} because the routers do not know the next optimal path if one link cost increases. This is known as the {\bf count-to-infinity problem}.
\item Under normal conditions, the distance vectors will converge to the actual least cost distance.
\end{itemize}

%%%%%%%%%%%%%%%%%%%%%%%%%%%%%%%%%%%
\subsection{Comparison of LS and DV algorithms}

\begin{itemize}
\item Message complexity:
\begin{itemize}
\item {\bf LS}: with n nodes and E links, $O(nE)$ messages are sent.
\item {\bf DV}: Exchanges are only done between neighbours. Thus the convergence time varies.
\end{itemize}
\item Speed of convergence:
\begin{itemize}
\item {\bf LS}: $O(n^2)$ algorithm requires $O(nE)$ messages.
\item {\bf EV}: Convergence time varies because there may be routing loops, and the count-to-infinity problem may occur.
\end{itemize}
\item Robustness (what happens if router malfunctions):
\begin{itemize}
\item {\bf LS}: Node can advertise incorrect link cost. It won't affect other nodes cause nodes only compute their own routing table.
\item {\bf EV}: DV nodes can advertise incorrect path cost. Each node's table is used by others, so it could cause errors throughout the network.
\end{itemize}
\end{itemize}























\end{document}