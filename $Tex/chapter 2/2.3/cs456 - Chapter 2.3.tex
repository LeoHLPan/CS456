\documentclass{article}

\usepackage[utf8]{inputenc}
\usepackage{fullpage}
\usepackage{times}
\usepackage{tcolorbox}
\usepackage{enumitem}
\usepackage{multicol}
\usepackage{bookmark}

\bookmarksetup{
  numbered, 
  open,
}

\setlist{itemsep=2pt}
\renewcommand{\thesection}{Chapter 2.\arabic{section}}
\renewcommand{\thesubsection}{2.\arabic{section}.\arabic{subsection}}
\renewcommand{\thesubsubsection}{2.\arabic{section}.\arabic{subsection}.\arabic{subsubsection}}
\setcounter{section}{2}

\begin{document}

\noindent
{CS 456 \hfill Hao Pan}\\
{Salahuddin, Mohammad}\\
{Spring 2018}

%%%%%%%%%%%%%%%%%%%%%%%%%%%%%%%%%%%

\begin{center}
\section{Electronic Mail}
\vspace{-4mm}
{\it SMTP, POP3, IMAP}\\
\noindent
{\hfill 31/05/2018 [Th]}
\end{center}

\subsection{Overview}

\begin{itemize}
\item Electronic mail has 3 major components:
\begin{itemize}
\item User agents
\item Mail servers
\item Simple mail transfer protocol: SMTP
\end{itemize}
\end{itemize}

\subsubsection{User Agent}

\begin{itemize}
\item Also known as {\it "mail reader"}.
\item Used for {\it composing, editing, and reading mail messages}.
\item {\bf Ex.} Outlook, thunderbird, iPhone mail client...
\item Incoming and outgoing messages are stored on the server.
\end{itemize}

\subsubsection{Mail Servers}

\begin{itemize}
\item The {\bf mail server} is the {\it mailbox} contains incoming messages for the user.
\item It contains a {\bf message queue} of outgoing messages that are ready to be sent.
\item {\it SMTP protocols} between mail servers in order to send email messages. The client sends mail servers while the server receives it.
\end{itemize}

\subsubsection{SMTP}

\begin{itemize}
\item TCP is used to reliably transfer email messages from client to server. {\bf Port 25} is used.
\item A {\bf direct transfer} is when mail is transferred from the {\it sending server} to the {\it receiving server}.
\item There are 3 phases of a transfer:
\begin{itemize}
\item Handshaking
\item Transfer of messages
\item Closure
\end{itemize}
\item Command and response interaction is similar to HTTP. Commands are in ASCII text, and the response consists of a {\it status code} and a {\it phrase}.
\item The message (header and body) must be in 7-bit ASCII.
\item SMTP uses persistent connections.
\item SMTP servers use CRLF.CRLF to determine the end of a message.
\end{itemize}

\subsubsection{Example Scenarios}

\begin{itemize}
\item See slide 2-52 for an example scenario.
\item See slide 2-53 for example SMTP interaction.
\end{itemize}

%%%%%%%%%%%%%%%%%%%%%%%%%%%%%%%%%%%
\subsection{Mail Access Protocols}

\begin{itemize}
\item {\bf SMTP} Delivers mail to the receiver's server and stores it.
\item There are several mail access protocols for mail retrieval from a server:
\begin{itemize}
\item {\bf POP}, or {\it Post Office Protocol}, where mail can be downloaded after authorization.
\begin{itemize}
\item POP3 is {\it stateless} across different sessions.
\item {\bf Download and delete} mode ensures that mail {\it cannot be re-read} if the client is changed.
\item {\bf Download and keep} mode keeps {\it copies} of messages across different clients. They will still exist in the mail server after download.
\end{itemize}
\item {\bf IMAP}, or {\it Internet Mail Access Protocol}. It has more features, such as manipulating messages stored on the server.
\begin{itemize}
\item All messages are kept on the server.
\item Users can organize messages in folders.
\item The user state is kept across sessions (such as the names of folders, and mapping between message ID and folder name).
\end{itemize}
\item {\bf HTTP}. Messages can be viewed and modified on an internet browser. {\it gmail, Hotmail, and Yahoo! Mail} are examples.
\end{itemize}
\item See slide 2-58 for an example usage of POP3 protocol.
\end{itemize}


















\end{document}