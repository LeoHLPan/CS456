\documentclass{article}

\usepackage[utf8]{inputenc}
\usepackage{fullpage}
\usepackage{times}
\usepackage{tcolorbox}
\usepackage{enumitem}
\usepackage{multicol}
\usepackage{bookmark}

\bookmarksetup{
  numbered, 
  open,
}

\setlist{itemsep=2pt}
\renewcommand{\thesection}{Chapter 2.\arabic{section}}
\renewcommand{\thesubsection}{2.\arabic{section}.\arabic{subsection}}
\renewcommand{\thesubsubsection}{2.\arabic{section}.\arabic{subsection}.\arabic{subsubsection}}
\setcounter{section}{4}

\begin{document}

\noindent
{CS 456 \hfill Hao Pan}\\
{Salahuddin, Mohammad}\\
{Spring 2018}

%%%%%%%%%%%%%%%%%%%%%%%%%%%%%%%%%%%

\begin{center}
\section{P2P Applications}
\noindent
{\hfill 07/06/2018 [Th]}
\end{center}

\subsection{Pure P2P Architecture}

\begin{itemize}
\item There is no always-on server.
\item End systems directly communicate with each other.
\item Peers are \emph{intermittently} connected and changes IP addresses often.
\item {\bf Examples}:
\begin{itemize}
\item File distribution (BitTorrent)
\item Streaming (KanKan)
\item VoIP (Skype)
\end{itemize}
\end{itemize}

%%%%%%%%%%%%%%%%%%%%%%%%%%%%%%%%%%%
\subsection{File Distribution Time}

\begin{itemize}
\item How does P2P compare with client-server in terms of file distribution of a size $F$ file from one server to $N$ peers:
\item Note that peer upload and download capacity is a \emph{limited resource}.
\item Note: $u_s = $ server upload capacity.
\item In the client-server architecture:
\begin{itemize}
\item Server transmission must sequentially send N file copies.
\item The time to send one copy is $F/u_s$, so the time to send $N$ copies is $NF/u_s$.
\item Client must download the file copies.
\item If $d_{min}$ is the minimum client download rate, then the minimum client download time is $F/d_{min}$.
\item Thus, the time to distribute file $F$ to $N$ clients using the {\bf client-server approach} is: $D_{c-s} \ge max\{NF/u_s,F/d_{min}\}$
\end{itemize}
\item In the P2P architecture:
\begin{itemize}
\item Server transmission must upload at least one copy, which has time $F/u_s$.
\item Clients must each download the file copy, which has min download time $F/d_{min}$.
\item Clients download a combined $NF$ bits, meaning the max upload rate... (?)
\end{itemize}
\item See example graph of client-server vs. P2P on slide 2-80.
\end{itemize}

\clearpage

%%%%%%%%%%%%%%%%%%%%%%%%%%%%%%%%%%%
\subsection{BitTorrent}
\vspace{-3mm}
\emph{An example P2P file distribution service.}

\begin{itemize}
\item A file is divided into 256Kb chunks.
\item Peers in the torrent send and receive file chunks.
\item {\bf Swarm} is a group of peers exchanging chunks of a file.
\item {\bf Trackers} track peers who are participating in the swarm.
\item The process is usually as follows:
\begin{itemize}
\item Peers join the swarm. They have no chunks, but will receive them from other peers over time.
\item New peers register with the tracker to get a \emph{list of peers} and connects to a subset of them (their neighbours).
\item While downloading, the peer will upload chunks to other peers.
\item Peers have the option of \emph{changing the peers} it exchange chunks with.
\item {\bf Churn}: peers will often come and go.
\item Once a peer has the entire file, it could leave, or they could remain in the torrent.
\end{itemize}
\end{itemize}

\subsubsection{Requesting File Chunks}
\begin{itemize}
\item At any given time, different peers have \emph{different subsets} of file chunks.
\item Peer will sometimes ask other peers for list of chunks that they have and request the missing chunks (\emph{rarest first}).
\end{itemize}

\subsubsection{Sending Chunks (Tit-for-Tat)}
\begin{itemize}
\item Send chunks to (4) peers that are currently sending them chunks at the \emph{highest rate}.
\item Other peers are \emph{choked} and do not receive chunks from this peer.
\item The top (4) peers are re-evaluated after a given amount of time (10s).
\item After a period of time (30s) another peer is \emph{randomly selected} to start sending chunks to, thus \emph{unchoking} this peer.
\item This newly selected peer joins the top (4).
\item See example of unchoking on slide 2-84.
\end{itemize}



























\end{document}