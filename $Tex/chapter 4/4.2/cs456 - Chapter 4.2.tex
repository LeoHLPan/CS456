\documentclass{article}

\usepackage[utf8]{inputenc}
\usepackage{fullpage}
\usepackage{times}
\usepackage{tcolorbox}
\usepackage{enumitem}
\usepackage{multicol}
\usepackage{bookmark}

\bookmarksetup{
  numbered, 
  open,
}

\setlist{itemsep=2pt}
\renewcommand{\thesection}{Chapter 4.\arabic{section}}
\renewcommand{\thesubsection}{4.\arabic{section}.\arabic{subsection}}
\setcounter{section}{1}

\begin{document}

\noindent
{CS 456 \hfill Hao Pan}\\
{Salahuddin, Mohammad}\\
{Spring 2018}

%%%%%%%%%%%%%%%%%%%%%%%%%%%%%%%%%%%

\begin{center}
\section{What's Inside a Router}
\noindent
\end{center}

\subsection{Overview of Router Architecture}

\begin{itemize}
\item A high-level view of a generic router architecture can be found on {\bf slide 4-11}.
\item The center of the router contains a {\bf high-seed switching fabric}. It is connected bidirectionally to a {\bf routing processor}. The switching fabric is also connected to a number of {\bf router input ports} and {\bf router output ports} unidirectionally.
\item The routing processor is in the \emph{control plane} while the switching fabric and the I/O ports are in the \emph{forwarding plane}.
\end{itemize}

%%%%%%%%%%%%%%%%%%%%%%%%%%%%%%%%%%%
\subsection{Input Port Functions}

\begin{itemize}
\item The input port contains mechanism for three functions. They are lined up sequentially.
\item The {\bf line termination} is first. It receives bits. It is in the \emph{physical layer}.
\item The {\bf link layer protocol} is up next. It is in the \emph{data link layer}.
\item Finally is the lookup, forwarding, and queueing. Its goal is to complete input port processing at \emph{"line speed"}. If datagrams arrive faster than the forwarding rate into the switch fabric, the datagrams will be placed into a queue.
\item After acquiring the output port from the header, one of two forwarding methods is used:
\begin{itemize}
\item {\bf Destination-based forwarding}: Forward based only on the destination IP address. This is the traditionally-used method.
\item {\bf Generalize forwarding}: Forward based on any set of header field values.
\end{itemize}
\item A chart showing destination-based forwarding is shown on {\bf slide 4-14}.
\item {\bf Longest prefix matching} is used when looking for a destination address. That is, use the \emph{longest} address prefix that matches the destination address. An example can be seen on {\bf slide 4-15}.
\item Longest prefix matching is often done using \emph{ternary content addressable memories (TCAMs)}.
\end{itemize}

%%%%%%%%%%%%%%%%%%%%%%%%%%%%%%%%%%%
\subsection{Switching Fabrics}

\begin{itemize}
\item The {\bf switching fabric} transfers packets from the input buffer to an appropriate output buffer.
\item {\bf Switching rate} refers to the rate at which packets can be transferred from inputs to outputs. It is often measured as a multiple of I/O line rate. For N inputs, a switching rate of N times line rate is desirable.
\item A visual of different switching fabrics can be seen on {\bf slide 4-17}.
\item If the fabric is slower than the input ports combined, then queueing may occur at the input queues, which may result in loss due to buffer overflow.
\end{itemize}

%%%%%%%%%%%%%%%%%%%%%%%%%%%%%%%%%%%
\subsection{Output Ports}

\begin{itemize}
\item The output port uses the same three functions as the input port (also sequentially), but in reverse order.
\item Buffering will happen if datagrams arrive from the fabric faster than the transmission rate, which could, once again, cause packet loss.
\item {\bf Scheduling discipline} chooses among the queued datagrams the best one to use for transmission.
\item The \emph{RFC 3439 rule of thumb} is that the average buffering is equal to "typical" RTT times link capacity C. With N flows, the buffering is equal to $\frac{RTT\cdot C}{\sqrt{N}}$.
\end{itemize}

%%%%%%%%%%%%%%%%%%%%%%%%%%%%%%%%%%%
\subsection{Scheduling Mechanisms}

\begin{itemize}
\item {\bf Scheduling} chooses the next packet to send on link.
\item {\bf FIFO scheduling} sends packets in order of arrival to the queue.
\item {\bf Priority scheduling} sends the queued packet with the highest priority. Priority can be determined by the class of the packet, which depends on marking and other header info (such as \emph{IP source/destination, port numbers, etc...}).
\end{itemize}




















\end{document}