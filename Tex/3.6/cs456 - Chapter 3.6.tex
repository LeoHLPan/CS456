\documentclass{article}

\usepackage[utf8]{inputenc}
\usepackage{fullpage}
\usepackage{times}
\usepackage{tcolorbox}
\usepackage{enumitem}
\usepackage{multicol}
\usepackage{bookmark}

\bookmarksetup{
  numbered, 
  open,
}

\setlist{itemsep=2pt}
\renewcommand{\thesection}{Chapter 3.\arabic{section}}
\renewcommand{\thesubsection}{3.\arabic{section}.\arabic{subsection}}
\setcounter{section}{5}

\begin{document}

\noindent
{CS 456 \hfill Hao Pan}\\
{Salahuddin, Mohammad}\\
{Spring 2018}

%%%%%%%%%%%%%%%%%%%%%%%%%%%%%%%%%%%

\begin{center}
\section{Principles of Congestion Control}
\noindent
\end{center}

\subsection{What is Congestion}

\begin{itemize}
\item Informally, {\bf congestion} is when sources are sending \emph{too much data} in a \emph{short} period of time such that the network can't handle all the data in time.
\item This is different from flow control.
\item Congestion could result in: \emph{lost packets} or \emph{very long delays}.
\item {\bf Note}: There are a lot of visuals. See {\bf slides 3-86 to 3-93}.
\end{itemize}

%%%%%%%%%%%%%%%%%%%%%%%%%%%%%%%%%%%
\subsection{Approaches Towards Congestion Control}

\begin{itemize}
\item Two approaches are typically used for congestion control.
\item End-end congestion control:
\begin{itemize}
\item The network does not provide explicit feedback.
\item End-systems infer congestion based on loss and delay.
\item This is the approach used by TCP.
\end{itemize}
\item Network-assisted congestion control:
\begin{itemize}
\item Routers provide feedback to end systems.
\item They provide a single bit to indicate the congestion level, and an explicit rate for the sender to send data at.
\end{itemize}
\end{itemize}

\end{document}