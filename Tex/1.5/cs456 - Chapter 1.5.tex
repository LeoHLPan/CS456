\documentclass{article}

\usepackage{fullpage}
\usepackage{times}
\usepackage[utf8]{inputenc}
\usepackage[english]{babel}
\usepackage{enumitem}

\usepackage{amsmath}

\setlist{itemsep=2pt}

\begin{document}

\noindent
{CS 456 \hfill Hao Pan}\\
{Salahuddin, Mohammad}\\
{Spring 2018}

%%%%%%%%%%%%%%%%%%%%%%%%%%%%%%%

\begin{center}
\underline{\large \bf Chapter 1.5 - Protocol Layers and Service Models}\\
\noindent
{\hfill 15/05/2018 [T]}
\end{center}

\underline{Protocol "Layers"}

\begin{itemize}
\item Networks are complex. They have many layers, including: {\it hosts, routers, links, applications, protocols, hardware, software, etc.}.
\item Each {\bf layer} implements a service. The layers have their own actions, but they also rely on services provided by other layers.
\item Advantages of layering:
\begin{itemize}
\item Allows {\it fast identification} and shows {\it relationship} between parts of a complex system.
\item Modularization makes maintenance easier. Changes to one layer does not affect the system, and the other layers do not know that a layer has been changed.
\end{itemize}
\end{itemize}

\underline{Internet Protocol Stack}

\begin{itemize}
\item {\bf Applications} support network applications. {\bf ex.} {\it FTP, SMTP, HTTP}.
\item {\bf Transport} handles process-process transfers. {\bf ex.} {\it TCP, UDP}.
\item {\bf Network} routes data from source to destination. {\bf ex.} {\it IP, routing protocols}.
\item {\bf Link} transfers data between neighbouring network elements. {\bf ex.} {\it Ethernet, 802.111, PPP}.
\item {\bf Physical} bits on the wire.
\end{itemize}

\underline{Encapsulation}

\begin{itemize}
\item See slide 1-62.
\end{itemize}

















\end{document}