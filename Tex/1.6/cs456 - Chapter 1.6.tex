\documentclass{article}

\usepackage{fullpage}
\usepackage{times}
\usepackage[utf8]{inputenc}
\usepackage[english]{babel}
\usepackage{enumitem}

\usepackage{amsmath}

\setlist{itemsep=2pt}

\begin{document}

\noindent
{CS 456 \hfill Hao Pan}\\
{Salahuddin, Mohammad}\\
{Spring 2018}

%%%%%%%%%%%%%%%%%%%%%%%%%%%%%%%

\begin{center}
\underline{\large \bf Chapter 1.6 - Networks Under Attack: Security}\\
\noindent
{\hfill 15/05/2018 [T]}
\end{center}

\underline{Network Security}\\

\vspace{-4mm}
{\it The internet was not originally designed with much security in mind.}

\begin{itemize}
\item The internet was originally envisoned to be "a group of mutually trusting users attached to a transparent network".
\item Obviously, the internet did not turn out that way. Bad people may try to {\it attack} computer networks.
\item Thus, internet protocol designers are playing "catch-up" in order to {\it defend} networks against attacks.
\item {\bf Malware} is placed into hosts via the internet. They can get in from:
\begin{itemize}
\item {\bf Virus}, which are self-replicating infections from receiving or executing an object ({\bf ex.} {\it e-mail attachment}).
\item {\bf Worm}, self-replicating infections from receiving an object that is executed on its own.
\end{itemize}
\item {\bf Spyware} can record keystrokes or websites visited. Collected info is uploaded to a collection site.
\item Infected hosts may be enrolled in a {\bf botnet}, which is used for {\it DDoS attacks}.
\item {\bf DoS} ({\it Denial of Service}) attacks make resources (such as server or bandwidth) unavailable by {\it legitimate traffic}. It does this by flooding the resources with {\it bogus traffic}. It uses the following steps:
\begin{itemize}
\item Hosts around the network are broken into and enrolled in the botnet.
\item Compromised hosts send packets to the target.
\end{itemize}
\item {\bf Packet sniffing} is also sometimes used to obtain data. A network places itself between a source and destination network and reads all packets passed between them.
\item {\bf IP spoofing}: when packets are sent with a {\it fake IP address} in order to mask the identity of the sender or to impersonate another system.
\end{itemize}













\end{document}