\documentclass{article}

\usepackage[utf8]{inputenc}
\usepackage{fullpage}
\usepackage{times}
\usepackage{tcolorbox}
\usepackage{enumitem}
\usepackage{multicol}
\usepackage{bookmark}

\bookmarksetup{
  numbered, 
  open,
}

\setlist{itemsep=2pt}
\renewcommand{\thesection}{Chapter 3.\arabic{section}}
\renewcommand{\thesubsection}{3.\arabic{section}.\arabic{subsection}}
\setcounter{section}{6}

\begin{document}

\noindent
{CS 456 \hfill Hao Pan}\\
{Salahuddin, Mohammad}\\
{Spring 2018}

%%%%%%%%%%%%%%%%%%%%%%%%%%%%%%%%%%%

\begin{center}
\section{TCP Congestion Control}
\noindent
\end{center}

\subsection{Overview}

\begin{itemize}
\item TCP's approach is to \emph{increase the sender's transmission rate} (window size) until loss occurs.
\item {\bf Additive increase} is to increase \emph{cwnd} by 1 MSS every RTT until loss is detected.
\item {\bf Multiplicative decrease} is to cut \emph{cwnd} by half after loss is detected.
\end{itemize}

\subsubsection{Details}
\begin{itemize}
\item The sender limits transmission such that $LastByteSent - LastByteAcked \le cwnd$.
\item The TCP sending rate is roughly $\frac{cwnd}{RTT}bytes/sec$.
\end{itemize}

%%%%%%%%%%%%%%%%%%%%%%%%%%%%%%%%%%%
\subsection{TCP Slow Start}

\begin{itemize}
\item When a connection is established, increase output rate \emph{exponentially} until loss is detected.
\item Commonly, cwnd starts at 1 MSS and \emph{doubles} every RTT (every time an ACK is received).
\item Thus, the initial rate of transfer is low, but grows rapidly.
\end{itemize}

%%%%%%%%%%%%%%%%%%%%%%%%%%%%%%%%%%%
\subsection{TCP: Detecting and Reacting to Loss}

\begin{itemize}
\item If loss is indicated by {\it timeout}, set cwnd to 1 MSS and grow window exponentially until a threshold is reached. At that point, it will grow linearly. In other words, it repeats the \emph{TCP Slow Start} method until a threshold is reached.
\item If loss is indicated by 3 duplicate ACKs being received, it indicates that the network is capable of delivering at least \emph{some} segments. Thus, cwnd is \emph{cut in half} and then grows linearly. This method is known as {\bf TCP RENO}.
\item Assuming there is always data to send, the average window size is $\frac{3}{4}\frac{W}{RTT}bytes/sec$ where $W$ is the window size where loss occurs.
\end{itemize}

%%%%%%%%%%%%%%%%%%%%%%%%%%%%%%%%%%%
\subsection{TCP Fairness}

\begin{itemize}
\item If $N$ TCP sessions are competing for the same \emph{bottleneck link} with bandwidth $R$, each session should have an \emph{average rate} of $R/K$.
\item To show TCP fairness in an example, assume two sessions are competing for a link.
\begin{itemize}
\item When the link is being underutilized, both sessions will increase usage by the \emph{same constant amount} (additive increase).
\item When the link is being overutilized (loss), multiplicative decrease is done to both sessions. Note that the session consuming more bandwidth will \emph{lose more bandwidth} from this procedure.
\item After a multiplicative decrease, the sessions will begin doing additive increase again.
\item Over a long period of time, both sessions' bandwidth will converge to $R/2$.
\end{itemize}
\end{itemize}

%%%%%%%%%%%%%%%%%%%%%%%%%%%%%%%%%%%
\subsection{Explicit Congestion Notification (ECN)}

\begin{itemize}
\item If there is congestion in a router, the \emph{router will mark two bits} in the IP header (ToS field) to indicate congestion.
\item The congestion indication will be sent to the receiver, who, after seeing the indication, will set an \emph{ECE bit} on the ACK back to sender to notify them of congestion.
\end{itemize}

\end{document}