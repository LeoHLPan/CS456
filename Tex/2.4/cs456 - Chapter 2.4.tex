\documentclass{article}

\usepackage[utf8]{inputenc}
\usepackage{fullpage}
\usepackage{times}
\usepackage{tcolorbox}
\usepackage{enumitem}
\usepackage{multicol}
\usepackage{bookmark}

\bookmarksetup{
  numbered, 
  open,
}

\setlist{itemsep=2pt}
\renewcommand{\thesection}{Chapter 2.\arabic{section}}
\renewcommand{\thesubsection}{2.\arabic{section}.\arabic{subsection}}
\renewcommand{\thesubsubsection}{2.\arabic{section}.\arabic{subsection}.\arabic{subsubsection}}
\setcounter{section}{3}

\begin{document}

\noindent
{CS 456 \hfill Hao Pan}\\
{Salahuddin, Mohammad}\\
{Spring 2018}

%%%%%%%%%%%%%%%%%%%%%%%%%%%%%%%%%%%

\begin{center}
\section{DNS (Domain Name System)}
\noindent
{\hfill 05/06/2018 [T]}
\end{center}

\subsection{Overview}

\begin{itemize}
\item People have many identifiers. They have their {\it names}, their {\it social security number}, their {\it passport}... So do network devices.
\item Internet hosts and routers use a {\it 32 bit IP address} for addressing datagrams.
\item The {\it "name"}, its url, is used by humans.
\item {\bf DNS} is used to map between IP address and name.
\begin{itemize}
\item It is a {\it distributed database} implemented in a hierarchy of many namer servers.
\item {\it Application-layer protcols} such as {\it hosts} and {\it name servers} communicate to {\it resolve} names.
\end{itemize}
\end{itemize}

%%%%%%%%%%%%%%%%%%%%%%%%%%%%%%%%%%%
\subsection{DNS Services and Structure}

\subsubsection{DNS Services}
\begin{itemize}
\item Hostname to IP address translation.
\item Host aliasing.
\item Mail server aliasing.
\item Load distribution (multiple IP addresses could correspond to one name).
\end{itemize}

\subsubsection{DNS Structure}
\begin{itemize}
\item See slide 2-63 for example of a DNS hierarchy.
\item Client queries must start at the root DNS server and work their way down to the right DNS server to get the right IP address.
\item Local name servers contact {\bf root name servers} if they cannot resolve a name. The root name server will contact an {\it authoritative name server} if they don't know the name server either. The mapping, once acquired, is returned to the local name server.
\item {\bf Top Level Down}, or {\it TLD} servers are responsible for domains such as {\it com, org, net, edu...} as well as top-level country domains like {\it uk, fr, ca, jp...}.
\item {\bf Authoritative} DNS servers provide mappings only for hosts within an organization.
\item {\bf Local DNS name servers} do not belong strictly in a hierarchy.
\begin{itemize}
\item Every ISP usually has one, including companies and universities.
\item When the host makes a query, the query is sent to its local DNS server, which has a cache of recent mappings stored locally. They act as a proxy and forward the query into the hierarchy.
\end{itemize}

\clearpage

\item There are 2 methods for DNS name resolution:
\begin{itemize}
\item {\bf Iterated query}, when the contacted server replies with the name of the server to contact instead.
\item {\bf Recursive query}, when the burden of the name resolution is passed onto the contacted name server.
\end{itemize}
\item Once a name server learns a mapping, it will {\it cache} that mapping - though the cache entry will eventually {\it time out (TTL)}.
\item Cached entries may be {\it out-of-date}, so if the name host's IP address changes, the new IP address will not be known until the TTLs expire.
\end{itemize}

%%%%%%%%%%%%%%%%%%%%%%%%%%%%%%%%%%%
\subsection{DBS Records}

\begin{itemize}
\item DNS records are stored in the following {\it resource records (RR)} format: {\bf (name, value, type, ttl)}.
\item The entries represent different things for different {\bf types}:
\begin{itemize}
\item {\bf type=A}:
\begin{itemize}
\item {\bf name} is the hostname.
\item {\bf value} is the IP address.
\end{itemize}
\item {\bf type=NS}:
\begin{itemize}
\item {\bf name} is the domain (ex. foo.com).
\item {\bf value} is the hostname of the authoritative name server of the current domain.
\end{itemize}
\item {\bf type=CNAME}:
\begin{itemize}
\item {\bf name} is the alias name for the canonical real name.
\item {\bf value} is the canonical name.
\end{itemize}
\item {\bf type=MS}:
\begin{itemize}
\item {\bf value} is the name of the mailserver associated with the name.
\end{itemize}
\end{itemize}
\end{itemize}

%%%%%%%%%%%%%%%%%%%%%%%%%%%%%%%%%%%
\subsection{DNS Protocol Messages}

\begin{itemize}
\item {\it Queries} and {\it replies} both have the same message format.
\item A visual example can be seen on slide 2-71.
\item {\bf Identification}: a 16-bit number is used for the query. Replies use the same number.
\item {\bf Flag} states whether:
\begin{itemize}
\item The message is query or reply.
\item Recursion is desired.
\item Recursion is available.
\item Reply is authoritative.
\end{itemize}
\item The message body contains 32-bit sections for:
\begin{itemize}
\item Name and type fields for a query.
\item RRs in response to the query.
\item Records for authoritative servers.
\item Additional possibly helpful information.
\end{itemize}
\end{itemize}

\clearpage

%%%%%%%%%%%%%%%%%%%%%%%%%%%%%%%%%%%
\subsection{Inserting Records into DNS}

\begin{itemize}
\item To insert a new record into DNS, the name must be registered at the {\bf DNS registrar}.
\item The name and IP addresses of authoritative name servers must be provided.
\item Two RRs will be placed into the corresponding TLD server: One is {\it NS}, and one is {\it A}.
\item See slide 2-73 for an example.
\end{itemize}

%%%%%%%%%%%%%%%%%%%%%%%%%%%%%%%%%%%
\subsection{Attacking DNS}

\begin{itemize}
\item DDoS attacks:
\begin{itemize}
\item The root servers are bombarded with traffic, which isn't very successful because traffic is filtered, and local DNS servers cache IPs which enable them to bypass the root server.
\item If the TLD servers are bombarded, it is potentially more dangerous.
\end{itemize}
\item Redirect attacks:
\begin{itemize}
\item Man-in-middle, where queries are intercepted.
\item DNS poisoning, when bogus replies are sent to the DNS servers to cache.
\end{itemize}
\item Exploit DNS for DDoS, when queries with spoofed source addresses are sent to the target IP.
\end{itemize}













\end{document}